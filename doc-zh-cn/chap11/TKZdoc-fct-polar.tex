\documentclass[../main.tex]{subfiles}
\begin{document}
% \section{Courbes en coordonnées polaires}
\section{极坐标曲线}

 \hypertarget{tfpo}{}
% \begin{NewMacroBox}{tkzFctPolar}{\oarg{local options}\marg{$f(t)$}}
%   \emph{$f(t)$ est une expression utilisant la syntaxe de \tkzname{gnuplot}. }
%
% \medskip
% \begin{tabular}{lll}
%  \toprule
%  options             & exemple & explication                         \\
%  \midrule
% \TAline{$x(t)$,$y(t)$}{\tkzcname{tkzFctPar[0:1]}\{\tkzcname{t**3}\}\{\tkzcname{t**2}\}}{$x(t)=t^3$,$y(t)=t^2$ }
%  \bottomrule
% \end{tabular}
%
% Les options sont celles de \TIKZ.
%
% \begin{tabular}{lll}
% \toprule
% options             & défaut & définition                         \\
% \midrule
% \TOline{domain}{0:2*pi}{domaine de la fonction}
% \TOline{samples}{200}{nombre de points utilisés}
% \TOline{id} {tkzfonct}{permet d'identifier les noms des fichiers auxiliaires}
% \TOline{color}{black}{couleur de la ligne}
% \TOline{line width} {0.4pt}{épaisseur de la ligne}
% \TOline{style} {solid}{style de la ligne}
% \bottomrule
% \end{tabular}
%
% \medskip
% \emph{ \tkzname{gnuplot} définit  $\pi$ avec \tkzname{pi} et \tkzname{fp.sty} avec \tkzcname{FPpi}. Les valeurs qui déterminent le domaine sont évaluées par \tkzname{fp.sty}. Il est possible d'utiliser soit \tkzname{pi}, soit  \tkzcname{FPpi}.}
%  \end{NewMacroBox}
\begin{NewMacroBox}{tkzFctPolar}{\oarg{命令选项}\marg{$f(t)$}}
  \emph{用\tkzname{gnuplot}语法表示$f(t)$。}

\medskip
\begin{tabular}{lll}
 \toprule
 参数             & 样例 & 说明                         \\
 \midrule
\TAline{$x(t)$,$y(t)$}{\tkzcname{tkzFctPar[0:1]}\{\tkzcname{t**3}\}\{\tkzcname{t**2}\}}{$x(t)=t^3$,$y(t)=t^2$ } 
 \bottomrule
\end{tabular}

可以使用所有有效\TIKZ{}选项。

\begin{tabular}{lll}
\toprule
选项             & 默认值 & 含义                         \\
\midrule
\TOline{domain}{0:2*pi}{定义域}
\TOline{samples}{200}{采样点数}
\TOline{id} {tkzfonct}{函数id}
\TOline{color}{black}{颜色}
\TOline{line width} {0.4pt}{线宽}
\TOline{style} {solid}{线型}
\bottomrule
\end{tabular}

\medskip
\emph{
\tkzname{gnuplot}中定义了\tkzname{pi}用于表示$\pi$,
\tkzname{fp.sty}中定义了\tkzcname{FPpi}用于表示$\pi$,
值域由\tkzname{fp.sty}确定。
在代码中,既可以用\tkzname{pi},也可以用\tkzcname{FPpi}。}
 \end{NewMacroBox}


\subsection{极坐标函数曲线示例1}

$ \rho(t)= \cos(t)*\sin(t) $

\begin{tkzexample}[latex=8cm]
\begin{tikzpicture}[scale=0.75]
 \tkzInit [xmin=-0.5,xmax=0.5,
           ymin=-0.5,ymax=0.5,
           xstep=0.1,ystep=.1]
 \tkzDrawX   \tkzDrawY
 \tkzFctPolar[domain=-2*pi:2*pi]{cos(t)*sin(t)}
\end{tikzpicture}
\end{tkzexample}

\newpage
\subsection{极坐标函数曲线示例2}
$ \rho(t)= \cos(2*t)  $

\begin{center}
\begin{tkzexample}[]
\begin{tikzpicture}[scale=1.25]
   \tkzInit [xmin=-1,xmax=1,
             ymin=-1,ymax=1,
             xstep=.2,ystep=.2]
  \tkzDrawX   \tkzDrawY
  \tkzFctPolar[domain=0:2*pi]{cos(2*t)}
\end{tikzpicture}
\end{tkzexample}
\end{center}


 \newpage
 \subsection{极坐标心形}
% From Mathworld :  \url{http://mathworld.wolfram.com/HeartCurve.html}
来自: \url{http://mathworld.wolfram.com/HeartCurve.html}

 $\rho(t)= 2-2*\sin(t)+\sin(t)*\sqrt(|\cos(t)|)/(\sin(t)+1.4  $

\vspace{2cm}
\begin{center}
\begin{tkzexample}[]
\begin{tikzpicture}[scale=3]
	\tkzInit[xmin=-5,xmax=5,ymin=-5,ymax=5]
	\tkzFctPolar[domain     = -pi:pi,
	             samples    = 800,
	             ball color = red,
	             shading    = ball]%
	  {2-2*sin(t)+sin(t)*sqrt(abs(cos(t)))/(sin(t)+1.4)}
\end{tikzpicture}
\end{tkzexample}
\end{center}

 \subsection{极坐标函数曲线示例4}
 $\rho(t)= 1-sin(t)$

\vspace{2cm}
\begin{center}
\begin{tkzexample}[vbox]
\begin{tikzpicture}[scale=4]
 \tkzInit [xmin=-5,xmax=5,ymin=-5,ymax=5,xstep=1,ystep=1]
 \tkzFctPolar[domain=0:2*pi,samples=400]{ 1-sin(t) }
\end{tikzpicture}
\end{tkzexample}
\end{center}


   \newpage
\subsection{极坐标大麻曲线}
 来自: \url{http://mathworld.wolfram.com/CannabisCurve.html}

$ \rho(t)=(1+.9*\cos(8*t))*(1+.1*\cos(24*t))*(1+.1*\cos(200*t))*(1+\sin(t)) $

\begin{center}
\begin{tkzexample}[vbox]
\begin{tikzpicture}[scale=2.5]
  \tkzInit [xmin=-5,xmax=5,ymin=-5,ymax=5,xstep=1,ystep=1]
  \tkzFctPolar[domain=0:2*pi,samples=1000]%
  { (1+.9*cos(8*t))*(1+.1*cos(24*t))*(1+.1*cos(200*t))*(1+sin(t)) }
\end{tikzpicture}
\end{tkzexample}
\end{center}


\newpage
% \subsection{Scarabaeus  Curve}
\subsection{斯卡贝斯曲线}
% From mathworld : \url{http://mathworld.wolfram.com/Scarabaeus.html}
来自: \url{http://mathworld.wolfram.com/Scarabaeus.html}

$\rho(t)=1.6*\cos(2*t)-3*\cos(t) $

\vspace{2cm}
\begin{center}
\begin{tkzexample}[]
\begin{tikzpicture}[scale=2.5]
	\tkzInit [xmin=-5,xmax=5,ymin=-5,ymax=5,xstep=1,ystep=1]
	\tkzFctPolar[domain=0:2*pi,samples=400]{1.6*cos(2*t)-3*cos(t) }
	\end{tikzpicture}
	\end{tkzexample}
\end{center}


\end{document}
\endinput

