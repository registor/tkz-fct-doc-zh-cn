\documentclass[../main.tex]{subfiles}
\begin{document}
% \section{Macros pour définir  des surfaces }
\section{区域填充命令}

% Il s'agit par exemple de représenter la partie du plan comprise entre la courbe représentative d'une fonction, l'axe des abscisses et les droites
%  d'équation $x=a$ et $x=b$.
例如,可能通过着色表示一个函数曲线与坐标横轴、直线$x=a$和$x=b$构成的区域。

\subsection{\tkzcname{tkzDrawArea}或\tkzcname{tkzArea}命令:区域填充}  \hypertarget{tda}{}

% \begin{NewMacroBox}{tkzDrawArea}{\oarg{local options}}
% Les options sont celles de \TIKZ.
%
% \begin{tabular}{lll}
% \toprule
% options             & défaut & définition                         \\
% \midrule
% \TOline{domain}{-5:5}{domaine de la fonction}
% \TOline{with}{a}{référence de la fonction}
% \TOline{color}{200}{nombre de points utilisés}
% \TOline{opacity} {no defaut}{trnsparence}
% \TOline{style}{black}{couleur de la ligne}
% \end{tabular}
% \end{NewMacroBox}
\begin{NewMacroBox}{tkzDrawArea}{\oarg{命令选项}}
可以使用所有有效\TIKZ{}选项。

\begin{tabular}{lll}
\toprule
选项             & 默认值 & 含义                         \\
\midrule
\TOline{domain}{-5:5}{函数定义域}
\TOline{with}{a}{指定函数}
\TOline{color}{200}{颜色}
\TOline{opacity} {无}{透明度}
\TOline{style}{black}{线型}
\end{tabular}
\end{NewMacroBox}

\subsection{对数函数}

\begin{tkzexample}[]
\begin{tikzpicture}[scale=2]
 \tkzInit[xmin=0,xmax=3,xstep=1,
          ymin=-2,ymax=2,ystep=1]
 \tkzGrid
 \tkzAxeXY
 \tkzFct[domain= 0.4:3]{1./x}
 \tkzDefPointByFct(1)
 \tkzGetPoint{A}
 \tkzDefPointByFct(2)
 \tkzGetPoint{B}
 \tkzLabelPoints[above right](A,B)
 \tkzDrawArea[color=blue!30,
              domain = 1:2]
 \tkzFct[domain = 0.5:3]{log(x)}
 \tkzDrawArea[color=red!30,
              domain = 1:2]
 \tkzPointShowCoord(A)
 \tkzPointShowCoord(B)
 \tkzDrawPoints(A,B)
\end{tikzpicture}
\end{tkzexample}

\subsection{简单示例}
\begin{tkzexample}[]
  \begin{tikzpicture}[scale=1.75]
   \tkzInit[xmin=0,xmax=800,xstep=100,
            ymin=0,ymax=2000,ystep=400]
   \tkzGrid
   \tkzAxeXY
   \tkzFct[domain = 0:800]{(1./90000)*\x*\x*\x-(1./100)*\x*\x+(113./36)*\x}
   \tkzDefPoint(450,400){a}
   \tkzDrawPoint(a)
   \tkzDrawArea[color=orange!50, domain =0:450]
   \tkzDrawArea[color=orange!80, domain =450:800]
  \end{tikzpicture}
\end{tkzexample}

%<--------------------------------------------------------------------------->

\newpage
\subsection{填充样式}
\begin{tkzexample}[]
\begin{tikzpicture}[scale=2]
  \tkzInit[xmin=-3,xmax=4,ymin=-2,ymax=4]
  \tkzGrid(-3,-2)(4,4)
  \tkzDrawXY
  \tkzFct[domain = -2.15:3.2]{(2+\x)*exp(-\x)}
  \tkzDrawArea[pattern=north west lines,domain =-2:2]
  \tkzDrawTangentLine[draw,color=blue](0)
  \tkzDrawTangentLine[draw,color=blue](-1)
  \tkzDefPointByFct(2)  \tkzGetPoint{C}
  \tkzDefPoint(2,0){B}
  \tkzDrawPoints(B,C)  \tkzLabelPoints[above right](B,C)
  \tkzRep
\end{tikzpicture}
\end{tkzexample}
   %<--------------------------------------------------------------------------->

\newpage
% \subsection{Surface comprise entre  deux courbes \tkzcname{tkzDrawAreafg}}
\subsection{\tkzcname{tkzDrawAreafg}命令:填充两条函数曲线之间的区域}

\hypertarget{tdafg}{}
% \begin{NewMacroBox}{tkzDrawAreafg}{\oarg{local options}}
% Cette macro permet de mettre en évidence une surface délimitée par les courbes représentatives de deux fonctions. La courbe (a) doit être au-dessus de la courbe (b).
%
% \medskip
% \begin{tabular}{lll}
%  \toprule
%  options             & défaut & explication    \\
% \midrule
% \TOline{between} {a and b}{référence des deux courbes}
% \TOline{domain= min:max}{domain=-5:5}{Les options sont celles de \TIKZ.}
% \TOline{opacity} {0.5}{transparence}
% \bottomrule
% \end{tabular}
%
% \emph{L'option \tkzname{pattern} de \TIKZ\ peut être utile !  }
% \end{NewMacroBox}
\begin{NewMacroBox}{tkzDrawAreafg}{\oarg{命令选项}}
该命令用于填充两条函数曲线包围的区域,要求曲线(a)必须高于曲线(b)。

\medskip
\begin{tabular}{lll}
 \toprule
 选项             & 默认值 & 说明    \\
\midrule
\TOline{between} {a and b}{曲线名称}
\TOline{domain= min:max}{domain=-5:5}{定义域,是\TIKZ{}的选项}
\TOline{opacity} {0.5}{透明度}
\bottomrule
\end{tabular}

\emph{可以使用所有有效的\TIKZ{}填充样式\tkzname{pattern}选项。 }
\end{NewMacroBox}
%<--------------------------------------------------------------------------->

% \subsection{Surface comprise entre deux courbes en couleur}
\subsection{两条函数曲线之间的区域}
% Par défaut, la surface définie est comprise entre les deux premières courbes.
默认情况下,区域位于两条函数曲线之间。

\begin{tkzexample}[vbox]
 \begin{tikzpicture}[scale=1.5]
   \tkzInit[xmax=5,ymax=5]
   \tkzGrid  \tkzAxeXY
   \tkzFct[domain = 0:5]{x}
   \tkzFct[domain = 1:5]{log(x)}
   \tkzDrawAreafg[color  = orange!50,domain = 1:5]
 \end{tikzpicture}
\end{tkzexample}

%<--------------------------------------------------------------------------->
\newpage
\subsection{设置填充图案}

\begin{tkzltxexample}[]
\tkzDrawAreafg[between= a and b,pattern=north west lines,domain = 1:5]
\end{tkzltxexample}

\begin{center}
  \begin{tkzexample}[vbox]
  \begin{tikzpicture}[scale=.8]
    \tkzInit[xmax=5,ymax=5]
    \tkzGrid
    \tkzAxeXY
    \tkzFct[domain = 0:5]{x}
    \tkzFct[domain = 1:5]{log(x)}
    \tkzDrawAreafg[between= a and b,pattern=north west lines,domain = 1:5]
  \end{tikzpicture}
  \end{tkzexample}
\end{center}

%<--------------------------------------------------------------------------->
% \subsection{Surface comprise entre deux courbes avec l'option \tkzname{between}}
\subsection{\tkzname{between}选项}
% Attention à l'ordre des références dans l'option \tkzname{between}. Seule la partie de la surface (b) est au-dessus de (a) est représentée.
注意,\tkzname{between}的引用顺序,在此表示函数曲线(b)高于函数曲线(a)。

\begin{tkzexample}[latex=7cm]
\begin{tikzpicture}[scale=1.25]
 \tkzInit[ymin=-1,xmax=5,ymax=3]
 \tkzGrid
 \tkzAxeXY
 \tkzFct[domain = 0.5:5]{1/x}% courbe a
 \tkzFct[domain = 1:5]{log(x)}% courbe b
 \tkzDrawAreafg[between=b and a,
            color=magenta!50,
            domain = 1:4]
\end{tikzpicture}
\end{tkzexample}

%<--------------------------------------------------------------------------->
\newpage
% \subsection{Surface comprise entre deux courbes : courbes de Lorentz}
\subsection{两条Lorentz函数曲线之间的区域}
% Ici aussi, attention à l'ordre des références dans l'option \tkzname{between}.
在此,也要注意\tkzname{between}选项中的函数曲线引用顺序。

\begin{tkzexample}[vbox]
\begin{tikzpicture}[scale=1.25]
  \tkzInit[xmax=1,ymax=1,xstep=0.1,ystep=0.1]
  \tkzGrid
  \tkzAxeXY
  \tkzFct[color   = red,domain = 0:1]{(exp(\x)-1)/(exp(1)-1)}
  \tkzFct[color   = blue,domain = 0:1]{\x*\x*\x}
  \tkzFct[color  = green,domain = 0:1]{\x}
  \tkzDrawAreafg[between = c and b,color=purple!40,domain = 0:1]
  \tkzDrawAreafg[between = c and a,color=gray!60,domain = 0:1]
\end{tikzpicture}
\end{tkzexample}

%<--------------------------------------------------------------------------->
% \subsection{Mélange de style}
\subsection{混合样式}

\begin{tkzexample}[]
\begin{tikzpicture}[scale=2.5]
   \tkzInit[xmin=-1,xmax=4,ymin=0,ymax=5]
   \tkzGrid
   \tkzAxeXY
   \tkzFct[domain = -.5:4]{ 4*x-x**2+4/(x**2+1)**2}
   \tkzFct[domain = -.5:4]{x-1+4/(x**2+1)**2}
   \tkzDrawAreafg[color=green,domain = 1:4]
   \tkzDrawAreafg[pattern=north west lines,domain = -.5:1]
   \tkzRep
   \tkzText(2.5,4.5){$C_f$}
   \tkzText(2.5,1){$C_g$}
\end{tikzpicture}%
\end{tkzexample}

\newpage  %<--------------------------------------------------------------------------->
% \subsection{Courbes de niveaux}
% Le code est intéressant pour la définition des fonctions constantes  aux lignes 10 et 11.
\subsection{水平曲线}
注意代码第10行和第11行的常量函数定义。

\begin{tkzexample}[num]
\begin{tikzpicture}[scale=.75]
 \tkzInit[xmax=20,ymax=12]
 \tkzGrid[color=orange,sub](0,0)(20,12)
 \tkzAxeXY
 \tkzFct[samples=400,domain =0:8]{(32-4*x)**(0.5)}   % a
 \tkzFct[samples=400,domain =0:18]{(72-4*x)**(0.5)}  % b
 \tkzFct[samples=400,domain =0:20]{(112-4*x)**(0.5)} % c
 \tkzFct[samples=400,domain =2:20]{(152-4*x)**(0.5)} % d
 \tkzFct[samples=400,domain =12:20]{(192-4*x)**(0.5)}% e
 \def\tkzFctgnuf{0} % f
 \def\tkzFctgnug{12}% g
 \tkzDrawAreafg[between= b and a,color=gray!80,domain = 0:8]
 \tkzDrawAreafg[between= b and f,color=gray!80,domain = 8:18]
 \tkzDrawAreafg[between= d and c,color=gray!50,domain = 2:20]
 \tkzDrawAreafg[between= g and c,color=gray!50,domain = 0:2]
 \tkzDrawAreafg[between= g and e,color=gray!20,domain =12:20]
\end{tikzpicture}%
\end{tkzexample}

\end{document}
\endinput
