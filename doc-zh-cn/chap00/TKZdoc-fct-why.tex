\documentclass[../main.tex]{subfiles}
\begin{document}
% \section{Fonctionnement}
\section{概述}

% \TIKZ\ apporte différentes possibilités pour obtenir les représentations graphiques des fonctions. J'ai privilégié l'utilisation de \tkzname{gnuplot}, car je trouve \tkzname{pgfmath} trop lent et les résultats trop imprécis.
\TIKZ{}为绘制函数曲线提供了多种方式,
但其\tkzname{pgfmath}库太慢,且结果精确不高,
因此,使用\tkzname{gnuplot}工具绘制函数曲线是一个更为合理的选择。

% Avec \TIKZ\ et \tkzname{gnuplot}, on obtient la représentation d'une fonction à l'aide de
在\TIKZ{}中,使用\tkzname{gnuplot}工具\footnote{gnuplot是一个交互式命令行绘图工具,
需要提前安装该软件。}绘制函数曲线的基本语法是:

\begin{tkzltxexample}[]
  \draw[options] plot function {gnuplot函数表达式};
\end{tkzltxexample}

%  Dans cette nouvelle version de \tkzname{tkz-fct}, la macro \tkzcname{tkzFct} reprend le code précédent avec les mêmes options que celles de \TIKZ. Parmi les options, les plus importantes sont \tkzname{domain}  et \tkzname{samples}.
在新版\tkzname{tkz-fct}宏包中,用\tkzcname{tkzFct}命令及\TIKZ{}选项进行绘图。
其中,最重要选项是\tkzname{domain}和\tkzname{samples}。

% La macro \tkzcname{tkzFct} remplace \tkzcname{draw plot function} mais exécute deux tâches supplémentaires, en plus du tracé. Tout d'abord, l'expression de la fonction est sauvegardée avec la syntaxe de \tkzname{gnuplot} et également  sauvegardée avec la syntaxe de  \tkzname{fp} pour une utilisation ultérieure. Cela permet, sans avoir à redonner l'expression, de placer par exemple, des points sur la courbe (les images sont calculées à l'aide de  \tkzname{fp}), ou bien encore, de tracer des tangentes.
本宏包使用\tkzcname{tkzFct}命令代替了\tkzcname{draw plot function},
同时,还实现了两个重要的操作。
首先是用\tkzname{gnuplot}语法保存函数表达式,
同时,也用\tkzname{fp}保存该函数表达式,以方便后续使用。
这使得可以在不返回表达式的情况下,在曲线上布置点(通过\tkzname{fp}计算),
或者绘制切线。

% Ensuite, et c'est le plus important, \tkzcname{tkzFct} tient compte des unités utilisées pour l'axe des abscisses et celui des ordonnées. Ces unités sont définies en utilisant la macro \tkzcname{tkzInit} du package \tkzname{tkz-base} avec les options \tkzname{xstep} et \tkzname{ystep}.
当然,最重要的是,\tkzcname{tkzFct}命令能独立使用横坐标和纵坐标步长单位,
其步长单位由\tkzname{tzk-base}宏包的\tkzcname{tkzInit}命令,
通过\tkzname{xstep}和\tkzname{ystep}选项进行设置。

% La macro \tkzcname{tkzFct} intercepte les valeurs données à l'option \tkzname{domain} et évidemment l'expression mathématique de la fonction;
\tkzcname{tkzFct}命令可以使用\tkzname{domain}选项和函数表达式。
% si \tkzname{xstep} et \tkzname{ystep} diffèrent de 1 alors il est tenu compte de ces valeurs pour le domaine, ainsi que pour les calculs d'images. Lorsque \tkzname{xstep} diffère de 1 alors l'expression donnée, doit utiliser uniquement \tkzcname{x} comme variable, c'est ainsi qu'il est possible d'ajuster les valeurs.  Cela permet d'éviter des débordements dans les calculs.
如果\tkzname{xstep}和\tkzname{ystep}的步长为1,
则直接使用值域范围和计算结果绘制图像。
如果\tkzname{xstep}的步长不为1,
则表达式中的自变量须使用\tkzcname{x}宏表示,
这能有效避免计算中的数据溢出问题。

% Par exemple, soit à tracer le graphe de la fonction $f$  définie par :
例如,有如下函数$f$:

\[
f(x)=x^3 \text{其中} 0\leq x\leq 100
\]

% Les valeurs de $f(x)$ sont comprises entre 0 et $\numprint{1000000}$. En choisissant \tkzname{xstep=10} et \tkzname{ystep=100000}, les axes auront environ $10$ cm de longueur (sans mise à l'échelle).
$f(x)$的值域在$0$到$\numprint{1000000}$之间,
令\tkzname{xstep=10}和\tkzname{ystep=100000},
则坐标轴长度为$10$ cm(无缩放).

% Les valeurs du domaine seront comprises entre $0$ et $10$, mais l'expression donnée à \tkzname{gnuplot}, comportera des  \tkzcname{x} équivalents à $x \times 10$, enfin, la valeur finale  sera divisée par \tkzname{ystep=100000}. Les valeurs de $f(x)$ resteront ainsi  comprises entre $0$ et $10$.
这将可以将值域限定在$0$到$10$之间,
但\tkzname{gnuplot}表达式中的\tkzcname{x}则相当于$x \times 10$。
将计算结果除以\tkzname{ystep=100000},
可以使$f(x)$的结果不变,仍在$0$到$10$之间。

 \begin{tkzexample}[latex=10cm,small]
  \begin{tikzpicture}[scale=.5]
    \tkzInit[xmax=100,xstep=10,
             ymax=1000000,
             ystep=100000]
    \tkzDrawX[right]
    \tkzDrawY[above]
    \tkzLabelX[below=6pt]
    \tkzLabelY[left=6pt]
    \tkzGrid
    \tkzFct[color=red,
            domain=0:100]{\x**3}
  \end{tikzpicture}
\end{tkzexample}

\end{document}
\endinput
