\documentclass[../main.tex]{subfiles}
\begin{document}
% \section{Symboles}
\section{标记符号}
% Certains ajoutent aux courbes des symboles afin de donner des indications supplémentaires au lecteur. Voici quelques exemples possibles~:
可以通过为曲线添加不同标记符号以示区分,如:

\begin{tikzpicture}
\draw[thick,(-)](0,0)--(2,2);
\draw[thick,o-o](2,0)--(4,2);
\draw[thick,)-(](4,0)--(6,2);
\draw[thick,*-*](6,0)--(8,2);
  \end{tikzpicture}

\newcommand{\cred}[1]{{\color{red}#1}}
\newcommand{\cgreen}[1]{{\color{green!50!black}#1}}
\newcommand{\cblue}[1]{{\color{blue}#1}}

% L'exemple suivant est de \tkzname{Simon Schläpfer}~:
一个\tkzname{Simon Schläpfer}的例子:

可以分别标记分段函数不同的定义域的函数。
\[
y=\left\{\begin{array}{ll}
    \cred{8-1.5x}&,\text{if }x<2\\
    \cblue{4}&,\text{if }2 \leq x \leq 3\\
    \cgreen{2x-4}&,\text{if } x>3
  \end{array}
\right.
\]

\begin{center}
\begin{tkzexample}[vbox]
\begin{tikzpicture}
  \tkzInit[xmin=-1,xmax=6,ymin=0,ymax=10,xstep=1,ystep=1]
  \tkzGrid[color=gray]
  \tkzAxeXY
  \tkzFct[{-[},color=red,domain =-1:2,samples=2]{8-1.5*\x}
  \tkzFct[{[-]},color=blue,domain =2:3,samples=2]{4}
  \tkzFct[{]-},color=green!50!black,domain =3:6,samples=2]{2*\x-4}
\end{tikzpicture}
\end{tkzexample}

\end{center}


\end{document}
\endinput
