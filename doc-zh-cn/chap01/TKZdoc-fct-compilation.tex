\documentclass[../main.tex]{subfiles}
\begin{document}
% \section{Utilisation de Gnuplot}
\section{使用Gnuplot绘图}
%–––––––––––––––––––––––––––––––––––––––––––––––––––––––––––––––––––––––––––>
% \subsection{Mécanisme d'interaction entre \TIKZ\ et \tkzname{Gnuplot}}
\subsection{\TIKZ{}与\tkzname{Gnuplot}交互}

% \TEX\  est un système logiciel de composition de documents ( text processing programm ). Il permet bien sûr de calculer, mais avec des moyens limités. \TIKZ\ est ainsi limité par \TEX\ pour effectuer des calculs. Pour rappel ±16383.99999 pt est l'intervalle dans lequel \TEX\ stocke ses valeurs. Sachant que 1 cm est égal à 28.45274 pt, on s'aperçoit que \TEX\ ne  peut traiter que des dimensions inférieures à 5,75 mètres environ.
\TEX{}本质上是一个文本处理软件系统,虽然可以用\TEX{}进行计算,但其方法有限。
同样,受\TEX{}制约,\TIKZ{}的计算功能也不够完善。
由于\TEX{}的尺寸范围=±16383.99999pt,而1cm=28.45274pt,
所以\TEX{}只能处理小于5.75米范围内的数据。
看起来,这已经足够大了,
但在一系列的计算中,却很容易超出这一限制。

\bigskip
 \tkzActivOff
  \newcommand{\drawpage}[4]{%
  \begin{scope}[xshift=#1, yshift=#2,font=\footnotesize]
    \filldraw[fill=white!75!#4,draw=#4, very thin]%
   (0,0) -- (4.2,0) -- (4.2,4.85) --(3.21,5.84)-- (0,5.84) -- cycle;
   \fill[fill=#4,shade,top color=#4,bottom color=#4!40]%
       (3.21,5.84) -- ++(0,-0.99) -- ++(0.99,0) -- cycle;
    \path (2.1,2.97) node{#3};
  \end{scope}
}
%
% \begin{center}
% \begin{tikzpicture}[>=triangle 45,scale=.75]
% \drawpage{0cm}{0cm}{\texttt\blue\begin{minipage}{2cm}
% sample.tex
%
% with
%
% \tkzcname{draw plot[id=fct] function{---.};}
% \end{minipage}}{blue}
% \drawpage{12cm}{0cm}{\texttt \red sample.fct.gnuplot}{red}
% \drawpage{12cm}{-14cm}{\texttt\red sample.fct.table}{red}
% \drawpage{0cm}{-14cm}{\texttt\blue\begin{minipage}{2cm}
% sample.pdf
%
% \bigskip
% \shorthandoff{:}
%  \begin{tikzpicture}[domain=-1.5:.8]
%   \draw plot[id=f1,samples=200] function{x*x};
%  \end{tikzpicture}
% end{minipage}}{blue}
%
% \path (8.05,2.9) node(A)
%      [diamond,%
%       draw,color   = black,
%       fill         = red!60,%
%       text         = black,%
%       minimum size = 3 cm,%
%       font         = \normalsize]
%      {{\texttt \tikzname-\TEX}};
%   \path (14.1,-4.08) node(B)
%      [diamond,%
%       draw,color=black,fill=green!60,%
%       text = black,%
%       minimum size = 3 cm,%
%       font         = \normalsize]
%      {{\texttt gnuplot}};
%   \path (8.05,-11.1) node(C)
%      [diamond,%
%       draw,color   = black,
%       fill         = red!60,%
%       text         = black,%
%       minimum size = 3 cm,%
%       font         = \normalsize]
%      {{\texttt \tikzname-\TEX}};
%   \draw[->] (4.2,2.9) -- (A.west);
%   \draw[->] (A.east) -- (12,2.9);
%   \draw[->] (14.1,0) -- (B.north);
%   \draw[->] (B.south) -- (14.1,-8.18);
%   \draw[->] (12 ,-11.1)--(C.east);
%   \draw[->] (C.west)--(4.2,-11.1);
%   \draw[->,magenta] (4.2,2.9) to [ out =-80,in=260] node[below,pos=.5]{étape 1} (12,2.9);
%   \draw[->,magenta] (14.1,0) to [ out =200,in=160] node[left,pos=.5]{étape 2} (14.1,-8.18);
%   \draw[->,magenta] (12 ,-11.1) to [ out =110,in=70] node[above,pos=.5]{étape 3} (4.2,-11.1);
%   \end{tikzpicture}
% \end{center}
\begin{center}
\begin{tikzpicture}[>=triangle 45,scale=.75]
\drawpage{0cm}{0cm}{\texttt\blue\begin{minipage}{2cm}
带有

\tkzcname{draw plot[id=fct] function{---.};}
的\enquote{.tex}文件,如

sample.tex
\end{minipage}}{blue}
\drawpage{12cm}{0cm}{\texttt \red sample.fct.gnuplot}{red}
\drawpage{12cm}{-14cm}{\texttt\red sample.fct.table}{red}
\drawpage{0cm}{-14cm}{\texttt\blue\begin{minipage}{2cm}
sample.pdf

\bigskip
% \shorthandoff{:}
 \begin{tikzpicture}[domain=-1.5:.8]
  \draw plot[id=f1,samples=200] function{x*x};
 \end{tikzpicture}
\end{minipage}}{blue}

\path (8.05,2.9) node(A)
     [diamond,%
      draw,color   = black,
      fill         = red!60,%
      text         = black,%
      minimum size = 3 cm,%
      font         = \normalsize]
     {{\texttt \tikzname-\TEX}};
  \path (14.1,-4.08) node(B)
     [diamond,%
      draw,color=black,fill=green!60,%
      text = black,%
      minimum size = 3 cm,%
      font         = \normalsize]
     {{\texttt gnuplot}};
  \path (8.05,-11.1) node(C)
     [diamond,%
      draw,color   = black,
      fill         = red!60,%
      text         = black,%
      minimum size = 3 cm,%
      font         = \normalsize]
     {{\texttt \tikzname-\TEX}};
  \draw[->] (4.2,2.9) -- (A.west);
  \draw[->] (A.east) -- (12,2.9);
  \draw[->] (14.1,0) -- (B.north);
  \draw[->] (B.south) -- (14.1,-8.18);
  \draw[->] (12 ,-11.1)--(C.east);
  \draw[->] (C.west)--(4.2,-11.1);
  \draw[->,magenta] (4.2,2.9) to [ out =-80,in=260] node[below,pos=.5]{第1步} (12,2.9);
  \draw[->,magenta] (14.1,0) to [ out =200,in=160] node[left,pos=.5]{第2步} (14.1,-8.18);
  \draw[->,magenta] (12 ,-11.1) to [ out =110,in=70] node[above,pos=.5]{第3步} (4.2,-11.1);
  \end{tikzpicture}
\end{center}

\newpage

% Pour tracer des courbes en 2D en contournant ces problèmes, un moyen simple offert par \TIKZ, est d'utiliser \tkzname{gnuplot}.
\TIKZ{}通过简单地采用\tkzname{gnuplot}工具,从而绕过了这些限制,从而更加方便的绘制二维图像。

%  \tkzname{tkz-fct.sty}  s'appuie sur le programme \tkzname{gnuplot} et le package  \tkzname{fp.sty}. Le premier est utilisé pour obtenir une liste de points, et le second pour évaluer ponctuellement des valeurs.
\tkzname{tkz-fct.sty}宏包基于\tkzname{gnuplot}工具和\tkzname{fp.sty}宏包,用\tkzname{gnuplot}获取需要绘图点集,用\tkzname{fp.sty}宏包实现计算。

%  Vous devez donc installer \tkzname{Gnuplot},  son installation dépend de votre système, puis  il faudra que votre distribution trouve \tkzname{Gnuplot}, et que  \TeX\  autorise \tkzname{Gnuplot} à écrire un fichier.
因此,必须安装\tkzname{Gnuplot}工具,另外\LaTeX{}发行版要有访问\tkzname{Gnuplot}工具的权限,同时,\TeX{}也应该允许\tkzname{Gnuplot}能够写入需要的文件。关于如何安装\tkzname{Gnuplot}工具,请查阅相关资料,在此不再冗述。

% \begin{itemize}
% \item \textcolor{red}{\textbf{Étape 1}}
%
% On part du fichier \tkzname{sample.tex} suivant :
%
% \medskip
% \begin{tkzltxexample}[]
% \documentclass{article}
% \usepackage{tikz}
% \begin{document}
% \begin{tikzpicture}
% \draw plot[id=f1,samples=200,domain=-2:2] function{x*x};
% \end{tikzpicture}
% \end{document}
% \end{tkzltxexample}
%  \tkzActivOn
%
% La compilation de ce fichier créé avec \TIKZ, produit un fichier nommé    \tkzname{sample.f1.gnuplot}. Le nom du fichier est obtenu à partir de \tkzcname{jobname} et de l'option \tkzname{id}. Ainsi un même fichier peut créer plusieurs fichiers distincts. C'est un fichier texte ordinaire, affecté de l'extension \tkzname{gnuplot}. Il contient un préambule indiquant à \tkzname{gnuplot} qu'il doit créer une table contenant les coordonnées d'un certain nombre de points obtenu par la fonction définie par $x\longrightarrow x^2$. Ce nombre de points est défini par l'option \tkzname{samples}. Cette étape ne présente aucune difficulté particulière. Le fichier obtenu peut être traité manuellement avec \tkzname{gnuplot}.  Le résultat est le fichier suivant :
%
% \begin{tkzltxexample}[]
% set table; set output "sample.f1.table"; set format "%.5f"
% set samples 200; plot [x=-2:2] x*x
% \end{tkzltxexample}
%
% Une table sera créée et enregistrée dans un fichier texte nommé "sample.f1.table". Les nombres seront formatés pour ne contenir que 5 décimales.
% La table contiendra 201 couples de coordonnées.
%
% \item  \textcolor{red}{\textbf{Étape 2}}
%
% Elle est la plus délicate car  le fichier \tkzname{sample.f1.gnuplot} doit être ouvert par \tkzname{gnuplot}. Cela implique d'une part, que   \TEX\  autorise l'ouverture\footnote{c'est ici que l'on parle des options \tkzname{--shell-escape} et \tkzname{--enable-write18}}
%    du  fichier \tkzname{sample.f1.gnuplot} par \tkzname{gnuplot} et d'autre part, que   \TEX\ puisse trouver \tkzname{gnuplot}\footnote{c'est ici que l'on parle de \tkzname{PATH}}.
%
% Si \tkzname{gnuplot} trouve \tkzname{sample.f1.gnuplot} alors il produit un fichier texte \tkzname{sample.f1.table} si l'expression de la fonction est correcte.
%
%  \item \textcolor{red}{\textbf{Étape 3}}
%
%  Le fichier \tkzname{sample.f1.table} obtenu à l'étape précédente est utilisé par \TIKZ\ pour tracer la courbe.
%
% \medskip\hspace{1cm}
% \begin{tkzltxexample}[]
% # Curve 0 of 1, 201 points
% # Curve title: "x*x"
% # x y type
% -2.00000 4.00000  i
% -1.98000 3.92040  i
% -1.96000 3.84160  i
% ---.
% 1.98000 3.92040  i
% 2.00000 4.00000  i
% \end{tkzltxexample}
% \end{itemize}
\begin{itemize}
\item \textcolor{red}{\textbf{第1步}}

例如\tkzname{sample.tex}文件中的代码如下:

\medskip
\begin{tkzltxexample}[]
\documentclass{article}
\usepackage{tikz}
\begin{document}
\begin{tikzpicture}
\draw plot[id=f1,samples=200,domain=-2:2] function{x*x};
\end{tikzpicture}
\end{document}
\end{tkzltxexample}
 \tkzActivOn

编译这一段\TIKZ{}代码,会生成一个\tkzname{sample.f1.gnuplot}文件,
该文件的文件名由当前工作文件名\tkzcname{jobname}和\tkzname{id}选项值组合而成。
因此,一个文件可能会生成多个不同的\tkzname{.gnuplot}文件。
以\tkzname{gnuplot}为扩展名的文件是一个普通的纯文本文件,
该文件中包含一条命令,用于通知\tkzname{gnuplot}工具创建点坐标表文件,并且为坐标点命名,设置坐标点精度。
同时,还用\tkzname{samples}选项设置了由$x\longrightarrow x^2$定义的函数的采样点个数,
当然,也可以手动编写该\tkzname{gnuplot}文件:

\begin{tkzltxexample}[]
set table; set output "sample.f1.table"; set format "%.5f"
set samples 200; plot [x=-2:2] x*x
\end{tkzltxexample}

这两条命令能够创建一个点坐标构成的数据表,并保存到\enquote{sample.f1.table}文件中,
本例中,坐标包含5位小数,共有201个点。

\item  \textcolor{red}{\textbf{第2步}}

由于必须用\tkzname{gnuplot}工具打开\tkzname{sample.f1.gnuplot},
因此,一方面\TEX{}要允许\tkzname{gnuplot}打开\tkzname{sample.f1.gnuplot}
\footnote{请参阅\TEX{}的\tkzname{--shell-escape}和\tkzname{--enable-write18}命令行参数。},
另一方面\TEX{}要能找到\tkzname{gnuplot}工具\footnote{需要设置正确的\tkzname{PATH}环境变量。}。

如果\tkzname{gnuplot}能找到\tkzname{sample.f1.gnuplot}文件,
并且其函数表达式正确,则会生成\tkzname{sample.f1.table}坐标点数据表文件。

\item \textcolor{red}{\textbf{第3步}}

\TIKZ{}使用获得的\tkzname{sample.f1.table}数据表中的数据绘图。

% \medskip\hspace{1cm}
\begin{tkzltxexample}[]
# Curve 0 of 1, 201 points
# Curve title: "x*x"
# x y type
-2.00000 4.00000  i
-1.98000 3.92040  i
-1.96000 3.84160  i
---.
1.98000 3.92040  i
2.00000 4.00000  i
\end{tkzltxexample}
\end{itemize}
%
% \begin{enumerate}
%
% \item  Il faut remarquer qu'au cours d'une seconde compilation, si le fichier  \tkzname{sample.f1.gnuplot} ne change pas, alors \tkzname{gnuplot} n'est pas lancé et le fichier présent \tkzname{sample.f1.table} est utilisé.
%
% \item On peut aussi remarquer  que si vous êtes paranoïaque et que vous n'autorisez pas le lancement de gnuplot, alors une première compilation permettra de créer le fichier \tkzname{sample.f1.table}, ensuite manuellement, vous pourrez lancer gnuplot  et obtenir le fichier \tkzname{sample.f1.table}.
%
% \item Il est aussi possible de créer manuellement ou encore avec un quelconque programme, un fichier data.table que \TIKZ\ pourra lire avec
%
% \begin{tkzltxexample}[]
%   \draw plot[smooth] file {data.table};
% \end{tkzltxexample}
% \end{enumerate}
%
\begin{enumerate}

\item  注意如果\tkzname{sample.f1.gnuplot}时间戳没有变化,则不会再次启动\tkzname{gnuplot}工具,
	并且仍然使用当前\tkzname{sample.f1.table}数据表中的数据进行绘图。

\item 如果需要手动运行\tkzname{gnuplot},那么第一次编译后会创建\tkzname{sample.f1.table},
	并在后续处理中使用这个\tkzname{sample.f1.table}数据表文件。

\item 也可以手动或采用其它方式创建该数据表文件,然后供\TIKZ{}使用,如:

\begin{tkzltxexample}[]
  \draw plot[smooth] file {data.table};
\end{tkzltxexample}
\end{enumerate}
%
% \subsection{Installation de \tkzname{Gnuplot}}
\subsection{安装\tkzname{Gnuplot}}
%
% Gnuplot est proposé avec la plupart des distributions Linux, et existe pour OS X ainsi que pour Windows.
Gnuplot是一个跨平台的工具,在多数Linux发生版中已安装有该工具。当然,它也适用于OS X和Windows操作系统。
%
% \begin{enumerate}
%   \item \tkzname{Ubuntu}\NameSys{Linux Ubuntu} ou un autre système Linux: on l'installe en suivant la procédure classique d'installation d'un nouveau paquetage.
%   \item \tkzname{Windows}\NameSys{Windows XP}  Les utilisateurs de Windows doivent se méfier, après avoir téléchargé la bonne version et installé \tkzname{gnuplot} alors il faudra  renommé wgnuplot en gnuplot. Ensuite il faudra modifier le \tkzname{path}. Si le chemin du programme est \shorthandoff{:}\tkzname{C:\textbackslash gnuplot} alors il faudra ajouter   \tkzname{{C:}\textbackslash gnuplot\textbackslash bin\textbackslash}\shorthandon{:}  aux variables environnement (Aller à "Poste de Travail" puis faire "propriétés", dans l'onglet "Avancé", cliquer sur "Variables d'environnement". ).
% Ensuite pour compiler sous latex, il faudra ajouter au script de compilation l'option  \tkzname{--enable-write18 }.
%   \item  \tkzname{OS X}\NameSys{OS X}  C'est le système  qui pose le plus  de problème, car il faut compiler les sources.
% \end{enumerate}

\begin{enumerate}
  \item \tkzname{Ubuntu}:对于\NameSys{Ubuntu Linux}或其它Linux系统,按照常规软件包安装方式安装即可。
  \item \tkzname{Windows}:对于\NameSys{Windows}则需要注意下载与操作系统匹配的版本,
	  并在安装\tkzname{gnuplot}后,必须将\enquote{wgnuplot}重命名为\enquote{gnuplot}。
	  然后,还需要修改\tkzname{path}环境变量,例如:
	  如果程序安装在\tkzname{C:\textbackslash gnuplot}目录中,
	  则需要将\tkzname{{C:}\textbackslash gnuplot\textbackslash bin\textbackslash}添加到环境变量
	  (选择\enquote{计算机}的\enquote{属性},然后在\enquote{高级}选项卡中,
	  找到\enquote{环境变量}进行添加。)。
	  接下来,要注意,在后续的\LaTeX{}编译中,必须指定\tkzname{--enable-write18 }编译参数。
  \item  \tkzname{OS X}:对于\NameSys{OS X},则需要从源码编译后进行安装。
\end{enumerate}
%
% \subsection{ Test de l'installation de tkz-base}
% Enregister le code suivant dans un fichier avec le nom test.tex, puis compiler avec pdflatex ou bien lualatex. Vous devez obtenir cela :

\subsection{测试\tkzname{tkz-base}}
将以下代码保存为\enquote{test.tex}文件,然后用xelatex、pdflatex或lualatex进行编译:

\begin{tkzltxexample}[]
\documentclass{scrartcl}
\usepackage{tkz-fct}
\begin{document}
  \begin{tikzpicture}
      \tkzInit[xmin=-5,xmax=5,ymax=2]
      \tkzGrid
      \tkzAxeXY
  \end{tikzpicture}
\end{document}
\end{tkzltxexample}

  \begin{tikzpicture}
      \tkzInit[xmin=-5,xmax=5,ymax=2]
      \tkzGrid
      \tkzAxeXY
   \end{tikzpicture}

   如果能够绘制坐标网格和坐标轴,则表示\tkzname{tkz-base}宏包已正确安装。

% \subsection{ Test de l'installation de tkz-fct}
   \subsection{测试\tkzname{tkz-fct}}
% Il suffit d'ajouter une ligne pour tracer la représentation graphique d'une fonction.
仅需在上述代码中添加一句代码就可以绘制函数图形:

\begin{tkzltxexample}[]
\documentclass{scrartcl}
\usepackage[usenames,dvipsnames]{xcolor}
 \usepackage{tkz-fct}
  \begin{document}
    \begin{tikzpicture}[scale=1.25]
      \tkzInit[xmin=-5,xmax=5,ymax=2]
      \tkzGrid
      \tkzAxeXY
      \tkzFct[color=red]{2*x**2/(x**2+1)}
     \end{tikzpicture}
 \end{document}
\end{tkzltxexample}

\begin{tkzexample}[]
\begin{tikzpicture}[scale=1.25]
    \tkzInit[xmin=-5,xmax=5,ymax=2]
    \tkzGrid
    \tkzAxeXY
    \tkzFct[color=red]{2*x**2/(x**2+1)}
 \end{tikzpicture}
\end{tkzexample}

如果能够绘制函数图形,则表示\tkzname{tkz-fct}宏包已正确安装。

\end{document}
\endinput
