\documentclass[../main.tex]{subfiles}
\begin{document}
% \section{Droites particulières}
\section{特殊直线}
% \subsection{ Tracer une ligne verticale }
\subsection{垂线}
% \begin{NewMacroBox}{tkzVLine}{\oarg{local options}\marg{decimal number}}
% Attention, la syntaxe est celle de \tkzname{fp} car on n'utilise pas \tkzname{gnuplot} pour tracer une droite.
%
% \begin{tabular}{lll}
%   \toprule
% arguments &  exemple  & définition  \\
% \midrule
% \TAline{decimal number}{\tkzcname{tkzVLine\{1\}}}{Trace la droite $x=1$}
% \bottomrule
% \end{tabular}
%
% \medskip
% \begin{tabular}{lll}
% \toprule
% options  & défaut & définition             \\
% \midrule
% \TOline{color     }{|black| }{  couleur du trait}
% \TOline{line width}{|0.6pt| }{  épaisseur du point}
% \TOline{style     }{|solid|}{  style du trait }
% \bottomrule
% \end{tabular}
%
% \emph{voir les options les lignes dans \TIKZ}
% \end{NewMacroBox}
\begin{NewMacroBox}{tkzVLine}{\oarg{命令选项}\marg{decimal number}}
由于无法使用\tkzname{gnuplot}绘制直线,该命令需要使用\tkzname{fp}语法定义函数。

\begin{tabular}{lll}
  \toprule
参数 &  样例  & 含义  \\
\midrule
\TAline{decimal number}{\tkzcname{tkzVLine\{1\}}}{垂线$x=1$}
\bottomrule
\end{tabular}

\medskip
\begin{tabular}{lll}
\toprule
选项  & 默认值 & 含义             \\
\midrule
\TOline{color     }{|black| }{颜色}
\TOline{line width}{|0.6pt| }{线宽}
\TOline{style     }{|solid|}{线型}
\bottomrule
\end{tabular}

\emph{可以使用所有有效\TIKZ{}选项。}
\end{NewMacroBox}


% \subsection{Ligne verticale }
\subsection{绘制垂线}
% problème avec cette macro, en principe 1./3 devrait être acceptée.
注意使用带小数点的数,以实现浮点数计算。

\begin{tkzexample}[latex=8cm]
\begin{tikzpicture}[scale=2]
   \tkzInit[xmax=3,ymax=2]
   \tkzAxeXY
   \tkzVLine[color      = blue,
             style      = dashed,
             line width = 1pt]{2}
      \tkzVLine[color      = red,
             style      = dashed,
             line width = 1pt]{1./3}
\end{tikzpicture}
\end{tkzexample}


\newpage
% \begin{NewMacroBox}{tkzVLines}{\oarg{local options}\marg{list of values}}
% Attention, la syntaxe est celle de \tkzname{fp} car on n'utilise pas \tkzname{gnuplot} pour tracer une droite.
%
% \begin{tabular}{lll}
%   \toprule
% arguments &  exemple  & définition  \\
% \midrule
% \TAline{list of values}{\tkzcname{tkzVLines\{1,4\}}}{Trace les droites $x=1$ et $x=4$}
% \bottomrule
% \end{tabular}
%
% \end{NewMacroBox}
\begin{NewMacroBox}{tkzVLines}{\oarg{命令选项}\marg{list of values}}
由于无法使用\tkzname{gnuplot}绘制直线,该命令需要使用\tkzname{fp}语法定义函数。

\begin{tabular}{lll}
  \toprule
参数 &  样例  & 含义  \\
\midrule
\TAline{list of values}{\tkzcname{tkzVLines\{1,4\}}}{垂线$x=1$和$x=4$}
\bottomrule
\end{tabular}

\end{NewMacroBox}

% \subsection{Lignes verticales}
\subsection{绘制多条垂线}

\begin{tkzexample}[latex=7cm]
\begin{tikzpicture}
 \tkzInit[xmax=5,ymax=5]
 \tkzAxeXY
 \tkzVLines[color = green]{1,2,...,4}
\end{tikzpicture}
\end{tkzexample}

\subsection{用\tkzname{fp}计算垂线}
\begin{tkzexample}[]
\begin{tikzpicture}
  \tkzInit[xmin=-7,xmax=7,ymin=-1,ymax=1]
  \tkzAxeY
  \tkzAxeX[trig=2]
  \foreach\v in {-2,-1,1,2}
  {\tkzVLine[color=red]{\v*\FPpi}}
\end{tikzpicture}
\end{tkzexample}

\newpage
\subsection{水平线}
% \begin{NewMacroBox}{tkzHLine}{\oarg{local options}\marg{decimal number}}
% \begin{tabular}{lll}
% arguments &  exemple  & définition  \\
% \midrule
% \TAline{decimal number}{\tkzcname{tkzVLine\{1\}}}{Trace la droite $y=1$}
% \end{tabular}
% \end{NewMacroBox}
\begin{NewMacroBox}{tkzHLine}{\oarg{命令选项}\marg{decimal number}}
\begin{tabular}{lll}
参数 &  样例  & 含义  \\
\midrule
\TAline{decimal number}{\tkzcname{tkzVLine\{1\}}}{水平线$y=1$}
\end{tabular}
\end{NewMacroBox}

\begin{tkzexample}[latex=7cm]
\begin{tikzpicture}
  \tkzInit[xmax=80,xstep=20,ymax=2]
  \tkzAxeXY
  \tkzHLine[color=red]{exp(1)-1}
\end{tikzpicture}
\end{tkzexample}
% \subsection{Asymptote horizontale}
\subsection{水平渐近线}
% Attention, une autre méthode consiste à écrire \tkzcname{tkzFct{$\text{k}$}} mais si \tkzname{ystep= $n$} avec $n$ entier naturel alors il est nécessaire d'écrire $k$ comme un nombre réel, par exemple si \tkzname{ystep= $3$} alors il faut écrire $k=5.0$.
% 注意,另一种方法是使用\tkzcname{tkzFct{$\text{k}$}}命令。
% 如果\tkzname{ystep= $n$},带有$n$自然数整数。
% 则需要为$k$赋值,例如,如果\tkzname{ystep= $3$}则$k=5.0$。

\begin{tkzexample}[]
\begin{tikzpicture}[scale=2.5]
  \tkzInit[xmax=5,ymin=0.5,ymax=1.5,ystep=0.5]
  \tkzGrid
  \tkzAxeXY
  \tkzFct[domain = 0:10]{x*exp(-x)+1}
  \tkzHLine[color=red,style=solid,line width=1.2pt]{1}
  \tkzDrawTangentLine[draw,color=blue](1)
  \tkzText[draw,fill = brown!20](2,0.75){$f(x)=x \text{e}^{-x}+1$}
 \end{tikzpicture}
 \end{tkzexample}



\newpage

% \subsection{Lignes horizontales}
\subsection{绘制多条水平线}

% \begin{NewMacroBox}{tkzHLines}{\oarg{local options}\marg{list of values}}
% \begin{tabular}{lll}
% arguments &  exemple  & définition  \\
% \midrule
% \TAline{list of values}{\tkzcname{tkzHLines\{1,4\}}}{Trace les droites $y=1$ et $y=4$}
% \bottomrule
% \end{tabular}
% \end{NewMacroBox}
\begin{NewMacroBox}{tkzHLines}{\oarg{命令选项}\marg{list of values}}
\begin{tabular}{lll}
参数 &  样例  & 含义  \\
\midrule
\TAline{list of values}{\tkzcname{tkzHLines\{1,4\}}}{水平线$y=1$和$y=4$}
\bottomrule
\end{tabular}
\end{NewMacroBox}

\begin{tkzexample}[]
\begin{tikzpicture}
 \tkzInit
 \tkzAxeXY
 \tkzHLines[color = green]{1,2,...,10}
\end{tikzpicture}
\end{tkzexample}


\newpage
% \subsection{Asymptote horizontale et verticale}
\subsection{水平和垂直渐近线}
\begin{tkzexample}[vbox]
\begin{tikzpicture}[scale=1.25]
 \tkzInit
 \tkzGrid
 \tkzAxeXY
 \tkzFct[color=red,domain=1.001:1.9]{1+1/(log(x-1)**2)}
 \tkzFct[color=red,domain = 2.1:10]{1+1/(log(x-1)**2)}
 \tkzHLine[line width=1pt,color=red]{1}
 \tkzVLine[line width=1pt,color=blue]{2}
 \tkzDefPoint(1,1){A}
 \tkzDrawPoint[fill=white,color=Maroon,size=4](A)
 \tkzDefPointByFct[draw,with=b]({1+exp(1)})
 \tkzLabelPoint[above right](tkzPointResult){$(1+\text{e}~;~2)$}
 \tkzText[draw,color = black,fill = brown!20](6,6)%
          {$f(x)=\dfrac{1}{\ln^2 (x-1)}+1$}
\end{tikzpicture}
\end{tkzexample}


\end{document}
\endinput
